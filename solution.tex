
\chapter{Solution}
\label{cha:Solution}

To address these issues, we propose a method of distributed computing where functions become mobile instead of data. This way, data is isolated to each node, we avoid having to send sensitive user data, and we can reduce both network flow and amount of computation in certain scenarios. We implement this solution in the Elixir programming language and use \textit{pass-by-move} semantics to achieve this mobility.

In basically all programming languages, functions can be thought of as self contained procedures that take in some parameters and returns some value. So if we want to transport functions between nodes, we essentialy want a way to transport the instructions that define a function. Moreover, since we want to apply \textit{pass-by-move} semantics, we want this function to disappear from the sender once we can confirm that it has been sent away, while also making sure that the function is usable on the receiving end. 

For the implementation of this solution, we chose Elixir as the programming language. This language is built to work with the Erlang virtual machine, so we have all the distribution capabilities that are associated with Erlang. Additionally, Elixir is a functional programming language, so functions are first class objects. This means that functions are like any other variable, they can be passed into other functions in order to create more complex behavior. Futhermore, Elixir has metaprogramming capabilities, which allows the programmer to extend the language syntax using macros. Elixir's specialty in distributed computing, the fact that it's a functional language, and the capability of extending the language using metaprogramming makes it an ideal choice for the implementation of our solution.

In Elixir, all functions are defined within modules, and there can only exist one function with the same name and arity within the same module. This allows us to further define what our solution should look like. When sending a function, we need to specify the function name and arity, and the module where it lives. Once sent, this should dissapear from the functions defined by the sender's module and once received it should appear within the receiver's module. To achieve this, we have to consider Abstract Syntax Trees, or ASTs. 

An AST is a representation of a program's code in tree form. Usually, this representation is an intermediate step in the compilation process. Code, written in a text file format, is transformed into an AST which can then be used by the compiler to create an executable file. Crucially, AST's have all the required information necessary to run a program, and because of Elixir's metaprogramming capabilities, this AST is just like any other data structure within Elixir, ready to be examined and modified with other Elixir functions. So we have a more well defined path to our solution. First we must identify the the AST's of the modules of both the sender and receiver of a function. We must also identify the AST of the actual function. As established, this would have all the information that describes how to run the function. Moreover, the tree that describes the function would be a subtree of the AST corresponding to the module containing it. By removing this subtree from the module and compiling the resulting AST, we achieve the wanted behavior, a module which no longer has a certain function. We can send this function AST using Elixir primitives for distribution and once it is received all we have to do is insert it within the receiver's module AST. The resulting AST is compiled and once again we achieve the desired effect, a module with a new function, ready to be called.

This gives a very clear plan of action, now the question is how to implement it. Firstly, in order to capture the AST of an entire module, we have to use a macro. Macros are the basis of metaprogramming in Elixir. They work as a special sort of function that takes as input some code in AST representation and is expected to return some other code, also in AST representation. At compile time, macros are evaluated and the AST that they return is inserted directly into the final AST. So to capture the AST of an entire module, we define a macro that should be used in place of the traditional module defining syntax. This macro would return the AST of a traditional module, but includes a special function \textit{get-ast} that returns the AST of the module. This way, we have a way to access the module's AST during runtime. Moreover, by using a macro when defining these kinds of modules, the programmer can explicitly choose which modules will exhibit this behavior of mobile functions. Once this is done, inserting and deleting functions is a simple case of traversing the AST returned by our special function, making the necessary changes, and compiling the new, edited module, making sure to update the AST within the \textit{get-ast} function, to ensure that future changes will also work as expected.

This implements the basic functions necessary for the expected behavior of our solution, but we need a few more things in order to make this approach more comprehensive. Particularly, we want more functionalities to ensure the correct operation of our distributed system. Firstly, we want a way to identify whether or not a function exists within a module to ensure that we don't call nonexistent functions. This is a simple case of calling \textit{get-ast} and making sure the function we want is, in fact, part of the module. If it isn't we want a way to advertise to other nodes a certain function is needed. A simple \textit{send} is used to notify all nodes, and waiting for the function is done via \textit{wait-for-function}. This procedure involves entering a loop that processes incoming messages. If we receive the wanted function, we exit the loop and load the function. If we don't receive the function but instead receive a message advertising that another node needs some module, we check within our module to see if it exists and sends it if is the case, ignoring it otherwise. 

This collection of macros and functions gives us a comprehensive package for implementing distributed systems with the behavior we require. To use this library within our own application, all we have to do is list it as a dependency within the defacto package manager in Elixir and require the pacakge within the necessary files.