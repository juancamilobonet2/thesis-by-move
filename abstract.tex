% $Id:abstract.tex  $
% !TEX root = main.tex

\chapter{Abstract}

In a dsitributed computing system, functions usually live on a 
single machines, and data is shared between nodes in order to 
carry out calculations. This leads to undesirable characteristics,
such as a greater amount of shared information, increased network load,
and security issues. Fog computing challenges this standard, 
where processes can happen on edge devices. This work explores the 
pass-by-move semantics to distribute functions between many nodes
using Elixir's metaprogramming capabilities. To show the usability
of this approach, we implement a simple app that demostrates reduced
network load by XX\% and a reduced amount of shared data, which protects
data vulnerabilities and confidentiality.

% Computation in a distributed setting is usually 
% limited certain machines, and data is passed between 
% machines in order to carry out calculations. A recent
% paradigm is that of fog computing, where some calculations are 
% done on edge devices. The question arises on how to send and 
% receive instructions so that this type of computation can be 
% achieved. This work explores the use of pass-by-move semantics
% to distribute functions between nodes using Elixir's 
% metaprogramming capabilities. We show that 
% when dealing with large amounts of data, this form of distributed 
% computation can reduce execution time and network load.

\endinput

