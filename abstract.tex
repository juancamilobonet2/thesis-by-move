% $Id:abstract.tex  $
% !TEX root = main.tex

\chapter{Abstract}

In distributed computing systems, function entities live on single machines, and data is shared between nodes in order to 
carry out calculations. This leads to undesirable characteristics,
such as increased network load, a greater amount of shared information, and data vulnerabilities. The introduction of fog computing architectures, mitigates these problems by moving part of the computation to edge devices. To further tackle these problems, this work explores the use of pass-by-move semantics to distribute functions between many nodes, effectively reducing network load and avoiding unnecessary information sharing. To build the new semantics, we use  Elixir's metaprogramming capabilities. In order to show the usability of this approach, we implement a proof-of-concept app that demonstrates reduced network load and a reduced amount of shared data, which protects data vulnerabilities and confidentiality.

% Computation in a distributed setting is usually 
% limited certain machines, and data is passed between 
% machines in order to carry out calculations. A recent
% paradigm is that of fog computing, where some calculations are 
% done on edge devices. The question arises on how to send and 
% receive instructions so that this type of computation can be 
% achieved. This work explores the use of pass-by-move semantics
% to distribute functions between nodes using Elixir's 
% metaprogramming capabilities. We show that 
% when dealing with large amounts of data, this form of distributed 
% computation can reduce execution time and network load.

\endinput

