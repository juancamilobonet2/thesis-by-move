% $Id: introduction.tex 1784 2012-04-27 23:29:31Z nicolas.cardozo $
% !TEX root = main.tex

\chapter{Introduction}
\label{cha:introduction}



%Parrafo contexto
Distributed systems are a constant in today's world. Cloud computing, communications infrastructure, internet of things, and microservices are just a few examples of distributed systems in practice. Characteristic to these architectures is that functions permanently live on a single machine and data is moved around between computers in order to carry out tasks that are required of the system. For example, take two microservices deployed on different machines, one that deals with user authentication and another that handles requests based on the user's permissions. To successfully use each of the microservices, user data needs to move or communicate between them. Machines can communicate in many ways, like Rest APIs or remote procedure calls, but fundamentally only data is communicated or shared between processes.


There are several issues directly caused by the static nature of functions in a distributed system. Shared state is undesirable since it can lead to improper and unpredictable behaviors. 
% perhaps expand on this TODO
Moreover, when sending vast amounts of data for an operation that does not require much computational power, unnecessary strain is placed on the network, therefore it can be advantageous to perform the computations on the same device that has the information. Case in point, this is a core principle of fog computing, an architectural solution that values computation on edge devices rather than in the cloud when applicable. Data sharing is also troublesome when considering data security, as it gives attackers more opportunities to tap into and inspect sensitive data.

Taking into account the aforementioned problems existing in distributed computing, one might consider a model where functions are mobile between distributed machines rather than data. To do this, we need a way to pass functions between nodes as easily as one would pass regular data. Since functions encapsulate a sequence of instructions, what we require is to pass these instructions between nodes. Moreover, we must consider the mechanism by which these instructions are sent, and how the programmer interacts with this as part of a larger framework. 

In this work we present an implementation and validation of such a system, written in Elixir. In particular, we use pass-by-move semantics for function mobility between machines or distributed nodes. This implementation uses the metaprogramming capabilities of Elixir and can be adapted to any distributed system by making use of the defined macros and functions.

In order to validate the usability of the proposed solution, we present two implementations of a proof-of-concept distributed application using regular data sharing semantics, and using pass-by-move semantics for comparison of the implementations and their properties. The metrics used for this comparison are the network load on the entire system depending on how many petitions need to be processed. We show that when dealing with a larger amount of data, the proposed solution leads to lower network load than the traditional approach.


\endinput

